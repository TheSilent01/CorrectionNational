\documentclass{article}
\usepackage{amsmath} % for mathematical environments and symbols

\title{Correction National 2024}
\author{Yassine El Aidous}
\date{\today} % Use \today to automatically use the current date

\begin{document}

\maketitle % This line is important to add this so that you can write the title, author, and date

\begin{abstract}
    Ce document sera une proposition de correction de l'examen national.
\end{abstract}

\tableofcontents % It's for the sections created later on

\newpage

\section{Exercice 1 : Analyse}

On considère une fonction \( f \) définie sur \( [1, +\infty[ \) par :
\[
f(x) = 
\begin{cases}
    \frac{\ln x}{x^2 - 1} & \text{si } x > 1 \\
    \frac{1}{2} & \text{si } x = 1
\end{cases}
\]
Soit \( (C) \) la courbe représentative de la fonction \( f \) dans un repère orthogonal \( (O, \vec{\mathbf{i}}, \vec{\mathbf{j}}) \).

\subsubsection*{1- Montrer que \( f \) est continue à droite en 1}

Soit \( x \in ]1; +\infty[ \) % 'in' indicates membership

\[
\lim_{x \to 1^+} f(x) = \lim_{x \to 1^+} \frac{\ln x}{x^2 - 1} = \lim_{x \to 1^+} \frac{\ln x}{(x-1)(x+1)}
\]

\[
= \lim_{x \to 1^+} \frac{\ln x}{x-1} \cdot \frac{1}{x+1} = 1 \cdot \frac{1}{2} = \frac{1}{2}
\]

\( \text{car : } \lim_{x \to 1^+} \frac{\ln x}{x-1} = 1 \)

\subsubsection*{2 - Calculer \( \lim_{x \to +\infty} f(x) \) et interpreter graphiquement le resultat obtenu}
    Soit \( x \in ]1;+\infty[ \)
    \begin{align*}
        \lim_{x \to +\infty} f(x) &= \lim_{x \to +\infty} \frac{\ln x}{x^2 - 1} \\
        &= \lim_{x \to +\infty} \frac{\ln x}{x^2} \cdot \frac{1}{1 - \frac{1}{x^2}} \\
        &= 0 \cdot 1 \\
        &= 0
    \end{align*}

    \subsubsection*{Interpretation Graphique]}
    \begin{align*}
        & \text{La courbe } (C) \text{ admet une asymptote horizontale au voisinage de } +\infty \\
        & \text{de direction l'axe des abscisses.}
    \end{align*}
    

\end{document}
